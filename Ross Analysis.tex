\documentclass{article}
\usepackage[utf8]{inputenc}
\usepackage{amsmath}
\usepackage{amsthm}
\usepackage{amssymb}
\usepackage{natbib}
\usepackage{graphicx}
\usepackage[utf8]{inputenc}
\usepackage[english]{babel}
\usepackage{chngcntr}

\counterwithin*{equation}{section}
\counterwithin*{equation}{subsection}
\usepackage{hyperref}
\hypersetup{
    colorlinks=true,
    linkcolor=blue,
    filecolor=magenta,      
    urlcolor=cyan,
}
\title{Solution Manual of Elementary Analysis}
\author{Lu YuXun}
\date{February 2017}


\linespread{1.5}

\newtheorem{theorem}{Theorem}

\begin{document}
\maketitle
\section*{About This Solution Manual}
This solution manual is done by Yuxun LU from Nara Institute of Science and Technology. Corresponding textbook is ``Elementary Analysis The theory of Calculus" Second Edition by Kenneth A. Ross. This solution manual is under the Creative Commons License.
\section{Introduction}
\subsection{The Set of Natural Numbers}
\begin{proof}
\textbf{1.1}
We'll use induction to prove this. Our predicate is 
\\ $P(n): \sum_{i=1}^n i^2 = \frac{1}{6}n(n+1)(2n+1)$.
\\ \textit{Base case} $P(1): 1^2 = \frac{1}{6}(1 \times 2 \times 3)$.
\\ \textit{Inductive step} Assume $\forall n \in \mathbb{N}$, $P(n)$ is true. For $P(n+1)$,
\\ $1 + 2 + ... + n^2 + (n+1)^2
\\ = \frac{1}{6}n(n+1)(2n+1) + (n+1)^2
\\ = \frac{1}{6}(n+1)( n(2n+1) + 6(n+1) )
\\ = \frac{1}{6}(n+1)(2n^2 + 7n + 6) 
\\ = \frac{1}{6}(n+1)(n+2)(2n+3) 
\\ = \frac{1}{6}(n+1)(n+1+1)(2(n+1)+1)$
\\ $\therefore P(n) \implies P(n+1)$. Thus, $\forall n \in \mathbb{N}$, $P(n)$ is true.
\end{proof}
\begin{proof}
\textbf{1.2} We'll use mathematical induction to prove this. Our predicate is
\\ $P(n): \sum_{i=1}^n (8i-5) = 4n^2 - n$.
\\ \textit{Base case} $P(1) = 8 - 5 = 3 = 4 \times 1^2 - 1$.
\\ \textit{Inductive step} Assume $\forall n \in \mathbb{N}$, $P(n)$ is true. For $P(n+1)$,
\\ $\sum_{i=1}^{n+1} (8i-5)
\\ = 4n^2-n + 8(n+1) - 5 
\\ = 4n^2 + 7n + 3
\\ = 4n^2 + 8n + 4 - (n+1)
\\ = 4(n^2+2n+1) - (n+1)
\\ = 4(n+1)^2 - (n-1)
\\ \therefore P(n) \implies P(n+1)$. Thus, $\forall n \in \mathbb{N}$, $P(n)$ is true.
\end{proof}
\begin{proof}
\textbf{1.3} We'll use mathematical induction to prove this. Our predicate is
\\ $P(n): \sum_{i=1}^n i^3 = (\sum_{i=1}^n i)^2$.
\\ \textit{Base case} $P(1) = 1^3 = 1^2$.
\\ \textit{Inductive step} Assume $\forall n \in \mathbb{N}$, $P(n)$ is true. For $P(n+1)$,
\\ $\sum_{i=1}^{n+1} i^3 
\\ = (1 + 2 + ... + n )^2 + (n+1)^3
\\ = (1 + 2 + ... + n )^2 + (n+1)(n+1)(n+1)
\\ = (1 + 2 + ... + n )^2 + \frac{1}{2}(n+1)(n+1)(n+1) \cdot 2
\\ = (1 + 2 + ... + n )^2 + \frac{1}{2}( (n+1)n + n+1 )(n+1) \cdot 2
\\ = (1 + 2 + ... + n )^2 + 2 \cdot \frac{1}{2}(n+1)n(n+1) + \frac{1}{2}(n+1)^2 \cdot 2 
\\ = (1 + 2 + ... + n )^2 + 2(1 + 2 + ... + n)(n+1) + (n+1)^2
\\ = (1 + 2 + ... + n + n + 1)^2
\\ \therefore P(n) \implies P(n+1)$. Thus, $\forall n \in N$, $P(n)$ is true.
\end{proof}
\begin{proof}
\textbf{1.4} We'll use mathematical induction to prove our guess. Our guess is
\\ $P(n): \sum_{i=1}^n (2i-1) = n^2$.
\\ \textit{Base case} $P(1) = 1 = 1^2$.
\\ \textit{Inductive step} Assume $\forall n \in \mathbb{N}$, $P(n)$ is true. For $P(n+1)$,
\\ $\sum_{i=1}^{n+1} (2i-1) 
\\ = n^2 + 2(n+1) - 1
\\ = n^2 + 2n + 1
\\ = (n+1)^2
\\ \therefore P(n) \implies P(n+1)$. Thus, $\forall n \in \mathbb{N}$, $P(n)$ is true.
\end{proof}
\begin{proof}
\textbf{1.5} We'll use mathematical induction to prove this. Our predicate is
$\\ P(n): \sum_{i=0}^n \frac{1}{2^n} = 2 - \frac{1}{2^n}$
\\ \textit{Caution} $P(0)$ is a special case, which is true also. Since the definition of $\mathbb{N}$ on the textbook doesn't contains $0$, we start our proof with $P(1)$.
\\ \textit{Base case} $P(1) = \frac{3}{2} = 2 - 1 \frac{1}{2^1}$
\\ \textit{Inductive step} Assume $\forall n \in \mathbb{N}$, $P(n)$ is true. For $P(n+1)$,
\\ $2 - \frac{1}{2^n} + \frac{1}{2^{n+1}} = 2 - ( \frac{2-1}{2^{n+1}} ) = 2 - \frac{1}{2^{n+1}} \therefore P(n) \implies P(n+1)$. Thus, $\forall n \in \mathbb{N}, P(n)$ is true.
\end{proof}
\begin{proof}
\textbf{1.6} We'll use mathematical induction to prove this. Our predicate is
$\\ P(n): (11)^n - 4^n$ is divisible by $7$ when $n$ is a positive integer.
\\ \textit{Base case} $P(1) = 7 \because 7|7 \therefore P(1)$ is true.
\\ \textit{Inductive step} Assume $\forall n \in \mathbb{N}$, $P(n)$ is true. For $P(n+1)$,
\\ $11^{n+1} - 4^{n+1}
\\ = 11 \cdot 11^n - 4 \cdot 4^n
\\ = 10 \cdot 11^n - 3 \cdot 4^n + 11^n - 4^n
\\ = 10 \cdot (3 + 7 )^n - 3 \cdot 4^n + 11^n - 4^n
\\ = 10(t_0 7^n + t_1 4 \cdot 7^{n-1} + ... + t_{n-1}4^{n-1}7 + t_n 4^n$ where $t_i$ is the \textit{binominal coefficient}, $t_i = nCi$, $t_0 = t_n = 1$.
$\\= 10(t_0 7^n + t_1 4 \cdot 7^{n-1} + ... + t_{n-1}4^{n-1}7) + (10 - 3)4^n
\\ = 10(t_0 7^n + t_1 4 \cdot 7^{n-1} + ... + t_{n-1}4^{n-1}7) + 7 \cdot 4^n$.
\\ Since the first and the second terms are divisible by $7$, $P(n) \implies P(n+1)$. Thus, $\forall n \in \mathbb{N}$, $P(n)$ is true.
\end{proof}
\begin{proof}
\textbf{1.7} We'll use mathematical induction to prove this. Our predicate is
\\$\forall n \in \mathbb{N}$, $P(n): 7^n - 6n - 1$ is divisible.
\\ \textit{Base case} $P(1) = 0, \because 36 | 0, \therefore P(1)$ is true.
\\ \textit{Inductive step} Assume $\forall n \in \mathbb{N}$, $P(n)$ is true. For $P(n+1)$,
\\$7^{n+1} - 6(n+1) - 1
\\ = 7 \cdot 7^n - 6n - 6 - 1
\\ = 7 \cdot 7^n -6n - 7
\\ = 6 \cdot 7^n + 7^n - 6n - 7
\\ = 6(7^n - 1 ) + 7^n - 6 - 1
\\ = 6( (6+1)^n - 1 ) + 7^n - 6 - 1
\\ = 6( 6^n + t_1 6^{n-1} \cdot 1 + t_2 6^{n-2} \cdot 1^2 + ... + 1 - 1) + 7^n -6n - 1
\\ = ( 6^{n+1} + t_1 6^n \cdot 1 + t_2 6^{n-2} \cdot 1^2 + ... + t_{n-1}6^2) + 7^n -6n - 1$ where $t_i$ is the binominal coefficient.
\\ By our assumption we know the second term is divisible by $36 = 6^2$ and the first term is divisible by $36$. Therefore, $P(n) \implies P(n+1)$.
\\Thus, $\forall n \in \mathbb{N}$, $P(n)$ is true. 
\end{proof}
From now on the prove will be simplified.
\begin{proof}
\textbf{1.8} a) $P(2) = 4 > 3$.$\forall n \geq 2 \wedge n \in \mathbb{N}$, $P(n): n^2 > n + 1$.
\\ $P(n+1): (n+1)^2 = n^2 + 2n + 1 > (n+1) + 2n + 1 > (n+1) + 1$.
\\ b) $\forall n \geq 4 \wedge n \in \mathbb{N}$, $P(n): n! > n^2$. $P(n+1) = (n+1)! = (n+1)n! > (n+1)n^2> (n+1)^2$.
\end{proof}

\end{document}